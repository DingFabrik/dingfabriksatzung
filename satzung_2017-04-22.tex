\documentclass[12pt,ngerman]{scrartcl}
\usepackage{scrjura}
\usepackage[utf8]{inputenc}
\usepackage[T1]{fontenc}
\usepackage{babel}
\usepackage{csquotes}
\usepackage{microtype}

\setkomafont{section}{\sffamily\centering}

\usepackage{palatino}
\setlength{\parindent}{0pt}
\setlength{\parskip}{0.5em}

\begin{document}
\begin{center}
\Huge\textbf{Satzung des Dingfabrik Köln e.V.}\vspace*{1cm}

\Large{Stand 22.04.2017}
\end{center}\vspace*{1cm}

Die Gesellschaft unserer Tage entwickelt sich parallel in der Welt der Bits und der Welt der Atome: virtuell organisiert über das Internet und mit greifbaren Ergebnissen im Alltag stehen wir vor einem Umbruch, dessen konkrete Ausgestaltung zu einer der größten menschlichen und technischen Herausforderungen werden könnte. Die Dingfabrik Köln bietet Atomen und Bits gleichermaßen Raum zur Entfaltung und Vernetzung, strebt eine Demokratisierung des Produktionswissens an. Getragen vom schöpferischen Pioniergeist und kritischer Technikbegeisterung ist es Labor, Werkstatt, Begegnungsstätte und Lernort für die Denkenden und Handelnden, für die Schöpfer von Dingen, für Bastler und Tüftler. Die Türen der Dingfabrik stehen jedem offen, nicht nur Studierenden, Akademikern oder Besitzern von Spezialmaschinen, sondern auch ehrenamtlichen Unterstützern in pädagogischer, technischer oder finanzieller Hinsicht. Die Dingfabrik Köln bietet die Möglichkeit, eine zivile Idee für ein neues Produkt oder eine technische Entwicklung direkt in einen Prototypen umzusetzen und ist somit eine Fabrik, in der fast alles hergestellt und gelernt werden kann und darf.


\begin{contract}

\Clause{title={Name, Sitz, Geschäftsjahr}}

Der Verein führt den Namen \enquote{Dingfabrik Köln}. Der Verein wird in das Vereinsregister eingetragen und dann um den Zusatz \enquote{e.V.} ergänzt. Der Verein hat seinen Sitz in Köln.

Das Geschäftsjahr ist das Kalenderjahr. Das erste Geschäftsjahr beginnt mit der Eintragung des Vereins in das Vereinsregister und endet am 31.12. diesen Jahres.


\Clause{title={Zweck und Gemeinnützigkeit}}

Insbesondere fördert und unterstützt der Verein Vorhaben der Forschung, Wissenschaft und Volks- und Berufsbildung einschließlich der Studentenhilfe, sowie der Kunst und Kultur, internationaler Gesinnung, der Toleranz auf allen Gebieten, der Kultur und des Völkerverständigungsgedankens oder führt diese durch.

Der Vereinszweck soll unter anderem durch folgende Mittel erreicht werden:

\begin{itemize}
\item Bereitstellung von Arbeits- und Seminarräumen für Projektarbeit im Sinne des Satzungszwecks u.a. durch Einrichtung eines Versuchslabors
\item Anschaffung, Bereitstellung und Pflege sowie Weiterentwicklung von Arbeitsgeräten und Maschinen
\item Veranstaltung von Schulungen und Workshops zur Aus- und Weiterbildung an den Fertigungsmaschinen und Werkzeugen in den Vereinsräumen und in befreundeten Institutionen sowie in Handarbeitstechniken und allgemeinen Fertigungsverfahren inklusive der zugehörigen Werkstoffkunde
\item Veranstaltung von Vorträgen, Seminaren und Tagungen, auch und insbesondere zur Behandlung von offenen Fragen und aktuellen Entwicklungen in o.g. Themenbereichen
\item Projekte zur Förderung/Bildung/Erziehung in o.g. Bereichen, z.B. angeleitete Entwicklung und Gestaltung von Kunst- oder Designobjekten, Software-, Hardware- oder Elektronik-Komponenten, spezielle Bildungsveranstaltungen, Kooperation mit Bildungs- und Forschungseinrichtungen etc.
\item Durchführung von Bildungsveranstaltungen speziell für die Jugend
\item Vernetzung von bestehenden internationalen und regionalen Gruppen, z.B. User-Groups, Stammtische, Computerclubs, CoworkingSpaces, Künstlergruppen etc. sowie Durchführung von internationalen Kongressen und Konferenzen
\item Kontaktvermittlung zu bestehenden Gruppen
\end{itemize}

Einbindung künstlerischer Arbeiten zum Bereich Gesellschaft, Kultur, Fertigungs- und Handwerkstechniken, Computer, neue Medien in das Vereinsleben insbesondere durch- Ausstellung künstlerischer Arbeiten in den Vereinsräumen- Vorführung von Werkstücken, Modellen, Computerdemos u.ä.

Der Verein verfolgt ausschließlich und unmittelbar gemeinnützige Zwecke im Sinne des Abschnitts \enquote{Steuerbegünstigte Zwecke} der Abgabenordnung. Er darf keine Gewinne erzielen; er ist selbstlos tätig und verfolgt nicht in erster Linie eigenwirtschaftliche Zwecke. Die Mittel des Vereins werden ausschließlich und unmittelbar zu den satzungsgemäßen Zwecken verwendet. Die Mitglieder erhalten keine Zuwendung aus den Mitteln des Vereins. Niemand darf durch Ausgaben, die dem Zwecke des Vereins fremd sind oder durch unverhältnismäßig hohe Vergütungen begünstigt werden.


\Clause{title={Mitgliedschaft}}

Mitglieder des Vereins können natürliche und juristische Personen werden, die die Ziele des Vereins mittragen und unterstützen wollen. Es sind dabei folgende Arten von Mitgliedschaften vorgesehen:

\begin{itemize}
	\item Aktive Mitglieder sind natürliche Personen, die den Vereinszweck und die Verwirklichung der Vereinsziele durch Mitarbeit unterstützen und dabei die vollen Pflichten eines Vereinsmitglieds übernehmen. Insbesondere wird von ihnen Mitarbeit, die Teilnahme an den Mitgliederversammlungen und die Ausübung des Stimmrechts erwartet.
	\item Außerordentliche Mitglieder sind natürliche und juristische Personen, die durch ihre Mitgliedschaft im Verein die Unterstützung des Vereinszwecks und der Vereinsziele erklären, aber auf die Ausübung der Rechte der aktiven Mitglieder, hier die Ausübung des Stimmrechts auf der Mitgliederversammlung, verzichten. Juristische Personen benennen eine natürliche Person als Vertreter zur Ausübung der verbleibenden Rechte und Pflichten.
	\item Fördernde Mitglieder sind außerordentliche Mitglieder, die den Vereinszweck und die Vereinsziele insbesondere durch einen finanziellen oder Sachbeitrag fördern. Sie haben das Recht zur Teilnahme an der Mitgliederversammlung, ohne damit ein Stimmrecht zu erwerben.
\end{itemize}

Die Beitrittserklärung erfolgt in Textform gegenüber dem Vorstand. Über die Annahme der Beitrittserklärung entscheidet der Vorstand. Die Mitgliedschaft beginnt mit der Annahme der Beitrittserklärung.

Die aktive Mitgliedschaft wird auf Vorschlag eines aktiven Mitglieds mit Zustimmung zweier anderer aktiver Mitglieder durch Beschluss des Vorstands oder der Mitgliederversammlung mit einfacher Mehrheit erworben. Hauptentscheidungskriterium für die Aufnahme soll das von den Aufnahmekandidaten über einen längeren Zeitraum gezeigte Engagement und der dabei geleistete Beitrag im Sinne der Vereinsziele sein.

Ein aktives Mitglied kann auf eigenen Antrag beim Vorstand in die außerordentliche Mitgliedschaft wechseln. Bei Nichterfüllung der oben angegebenen Pflichten eines aktiven Mitglieds über zwei aufeinanderfolgende ordentliche Mitgliederversammlungen ändert sich die Mitgliedschaft automatisch in eine außerordentliche. Ein außerordentliches aber nicht förderndes Mitglied kann beim Vorstand die aktive Mitgliedschaft beantragen. Über die Aufnahme fördernder Mitglieder entscheidet der Vorstand.

Die Mitgliedschaft endet durch Austritt, Ausschluss oder Tod, bei juristischen Personen auch durch Verlust der Rechtspersönlichkeit.

\begin{itemize}
	\item Der Austritt eines Gründungsmitglieds ist frühestens sechs Monate nach Eintrag des Vereins ins Vereinsregister möglich. Danach ist der Austritt zum Monatsende des Folgemonats möglich. Er erfolgt in Textform gegenüber dem Vorstand.
	\item Der Austritt eines Mitglieds, ausgenommen Gründungsmitglieder, ist zum Monatsende des Folgemonats möglich. Er erfolgt in Textform gegenüber dem Vorstand.
\end{itemize}

Das Instrument des Vereinsausschlusses ist kritischen Situationen vorbehalten, wobei grundsätzlich der Klärung zur Güte der Vorrang zu gewähren ist. Der Ausschluss erfolgt auf Beschluss des Vorstands mit sofortiger Wirkung. Gründe für einen Ausschluss können sein

\begin{itemize}
	\item ein schwerer Verstoß eines Mitglieds gegen die in dieser Satzung festgelegten Bestimmungen sowie Ziele und Zwecke des Vereins nach einem erfolglosen Versuch der Klärung, sowie
	\item ein trotz mehrfacher Mahnung bestehender Rückstand an Beitragszahlungen über einen Zeitraum von 3 Monaten.Dem Mitglied muss vor der Beschlussfassung Gelegenheit zur Rechtfertigung bzw. Stellungnahme gegeben werden. Gegen den Ausschluss kann innerhalb von vier Wochen beim Vorstand Widerspruch eingelegt werden, über den die nächste Mitgliederversammlung entscheidet. Bis zur Entscheidung der Mitgliederversammlung ruhen die Rechte und Pflichten des Mitglieds.
\end{itemize}



Bei Ausscheiden eines Mitglieds aus dem Verein oder bei Vereinsauflösung besteht kein Anspruch auf Rückerstattung etwa eingebrachter Vermögenswerte.

\Clause{title={Rechte und Pflichten der Mitglieder}}

Die Mitglieder sind berechtigt, die Leistungen des Vereins in Anspruch zu nehmen.

Die Mitglieder sind verpflichtet, die satzungsgemäßen Zwecke des Vereins zu unterstützen und zu fördern. Sie sind verpflichtet, die festgesetzten Beiträge zu zahlen.

\Clause{title={Beitrag}}

Der Verein erhebt einen Aufnahme- und einen Jahresbeitrag. Das Nähere regelt eine Beitragsordnung, die von der Mitgliederversammlung beschlossen wird. Bei nicht fristgerechter Zahlung der Mitgliedsbeiträge ruht die Mitgliedschaft.

Im begründeten Einzelfall kann für ein Mitglied durch Vorstandsbeschluss ein von der Beitragsordnung abweichender Beitrag festgesetzt werden.

\Clause{title={Organe des Vereins}}

Die Organe des Vereins sind:

\begin{itemize}
	\item die Mitgliederversammlung
	\item der Vorstand.
\end{itemize}



\Clause{title={Mitgliederversammlung}}

Oberstes Beschlussorgan ist die Mitgliederversammlung. Ihrer Beschlussfassung unterliegen:

\begin{itemize}
	\item die Genehmigung des Finanzberichtes,
	\item  die Entlastung des Vorstandes,
	\item die Wahl und die Abberufung der Vorstandsmitglieder,
	\item die Bestellung von Finanzprüfern,
	\item die Satzungsänderungen,
	\item die Genehmigung der Beitragsordnung,
	\item die Richtlinie über die Erstattung von Reisekosten und Auslagen,
	\item Beschlüsse über Anträge des Vorstandes und der Mitglieder,
	\item die Ernennung von Ehrenmitgliedern,
	\item die Auflösung des Vereins und die Beschlussfassung über die eventuelle Fortsetzung des aufgelösten Vereins.
\end{itemize}

Die ordentliche Mitgliederversammlung findet jährlich statt. Außerordentliche Mitgliederversammlungen werden auf Beschluss des Vorstandes abgehalten, wenn die Interessen des Vereins dies erfordern, oder wenn mindestens 10\% der Mitglieder dies unter Angabe des Zwecks und der Gründe in Textform beantragen. Die Einberufung der Mitgliederversammlung erfolgt in Textform unter Angabe der Tagesordnung durch ein Vorstandsmitglied mit einer Frist von mindestens vier Wochen. Zur Wahrung der Frist genügt die Aufgabe der Einladung zur Post an die letzte bekannte Anschrift oder die Versendung an die zuletzt bekannte E-Mail-Adresse. Anträge zur Tagesordnung sind mindestens drei Tage vor der Mitgliederversammlung beim Vorstand einzureichen. Über die Behandlung von Initiativanträgen entscheidet die Mitgliederversammlung.

Beschlüsse über Satzungsänderungen und über die Auflösung des Vereins können nur in einer Mitgliederversammlung beschlossen werden, in der diese Tagesordnungspunkte ausdrücklich angekündigt worden sind. Solche Beschlüsse bedürfen zu ihrer Rechtswirksamkeit der Dreiviertelmehrheit der anwesenden stimmberechtigten Mitglieder. In allen anderen Fällen genügt die einfache Mehrheit.

Vorbehaltlich Absatz 3 bedürfen die Beschlüsse einer Mitgliederversammlung der einfachen Mehrheit der Stimmen der erschienenen stimmberechtigten Mitglieder.

Jedes Mitglied hat eine Stimme.

Die Mitgliederversammlung bestimmt einen Versammlungsleiter und einen Protokollführer.

Auf Antrag eines Mitglieds ist geheim abzustimmen. Über die Beschlüsse der Mitgliederversammlung ist ein Protokoll anzufertigen, das vom Versammlungsleiter und dem Protokollführer zu unterzeichnen ist; das Protokoll ist allen Mitgliedern zugänglich zu machen.

\Clause{title={Vorstand}}

Der Vorstand besteht aus zwei bis sieben Mitgliedern, und zwar:- dem Vorsitzenden,- einem oder zwei stellvertretenden Vorsitzenden,- bis zu vier weiteren Mitgliedern.

Vorstand im Sinne des §26, Abs. 2 BGB ist jedes Vorstandsmitglied. Ausgenommen sind Rechtsgeschäfte von über 50 Euro, Einstellung und Entlassung von Angestellten, gerichtliche Vertretung sowie Anzeigen, Aufnahme von Krediten, Gründung, Erwerb und Veräußerung von Gesellschaften und Geschäftsanteilen von Gesellschaften zur Verwirklichung der satzungsgemäßen Ziele, bei denen der Verein durch mindestens zwei Vorstandsmitglieder vertreten wird.

Die Amtsdauer der Vorstandsmitglieder beträgt zwei Jahre; Wiederwahl ist zulässig. Die gewählten Vorstandsmitglieder bleiben bis zu ihrer Amtsniederlegung oder Neuwahl im Amt.

Besteht der Vorstand aus weniger als zwei Mitgliedern, so sind unverzüglich Nachwahlen durchzuführen.

Beschlüsse des Vorstands werden mit der Mehrheit der Stimmen der an der Beschlussfassung teilnehmenden Vorstandsmitglieder gefasst. Bei Stimmengleichheit gibt die Stimme des Vorsitzenden, bei seiner Verhinderung die des an Lebensjahren ältesten stellvertretenden Vorsitzenden den Ausschlag. Der Vorstand ist Dienstvorgesetzter aller vom Verein angestellten Mitarbeiter; er kann diese Aufgabe einem Vorstandsmitglied übertragen.

Eine vom Vorstand bestimmte natürliche oder nicht-natürliche Person, die kein Vereinsmitglied sein muss, überwacht als Schatzmeister die Haushaltsführung und verwaltet unter Beachtung etwaiger Vorstandsbeschlüsse das Vermögen des Vereins. Er hat auf eine sparsame und wirtschaftliche Haushaltsführung hinzuwirken. Mit Ablauf des Geschäftsjahres stellt er unverzüglich die Abrechnung sowie die Vermögensübersicht und sonstige Unterlagen von wirtschaftlichem Belang den Finanzprüfern des Vereins zur Verfügung. Der Schatzmeister ist verpflichtet einen Bericht auf der ordentlichen Mitgliederversammlung vorzustellen, der keine persönlichen Daten enthalten darf.

Die Vorstandsmitglieder sind grundsätzlich ehrenamtlich tätig; sie haben Anspruch auf Erstattung notwendiger Auslagen im Rahmen einer von der Mitgliederversammlung zu beschließenden Richtlinie über die Erstattung von Reisekosten und Auslagen.

Der Vorstand gibt sich eine Geschäftsordnung, die von der Mitgliederversammlung zu genehmigen ist.


\Clause{title={Finanzprüfer}}

Zur Kontrolle der Haushaltsführung bestellt die Mitgliederversammlung einen oder zwei Finanzprüfer. Nach Durchführung ihrer Prüfung geben sie dem Vorstand Kenntnis von ihrem Prüfungsergebnis und erstatten der Mitgliederversammlung Bericht.

Die Finanzprüfer dürfen dem Vorstand nicht angehören.

\Clause{title={Auflösung des Vereins}}

Bei der Auflösung des Vereins oder bei Wegfall seines Zwecks fällt das Vereinsvermögen an eine von der Mitgliederversammlung zu bestimmende steuerbegünstigte Körperschaft, die das Vermögen zur Förderung der Forschung, Wissenschaft und Bildung zu verwenden hat.


\end{contract}



\end{document}
